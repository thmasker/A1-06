\documentclass[a4paper, spanish]{article}
\usepackage[utf8x]{inputenc}
\usepackage{indentfirst}
\usepackage[colorlinks=true,linkcolor=blue,urlcolor=blue]{hyperref}

\date{\textbf{\url{https://github.com/thmasker/A1-06}}}
\author{Pablo Alcázar Morales \and Güray Meriç \and Diego Pedregal Hidalgo}
\title{\textbf{OSM Project \textbar A1-06}}
\begin{document}
	\maketitle
	\section{Graph.py - \textit{Task1}}
	For this task, we found a simple, effective and easy to use library on Python, called  \href{https://docs.python.org/3/library/xml.etree.elementtree.html#module-xml.etree.ElementTree}{\textit{\underline{ElementTree}}}, so after seeing other options like \textit{minidom}, we decided to use it.
	After that, we created the required \textit{Graph} class, which constructor parses the chosen \textit{.graphml} file, and stores both nodes and edges in two independent variables.
	
	
	The complex of the three created methods is \textit{adjacentNode}: we go through the inner edges data and store all the specified information about the desired node's adjacents (source node, target node, street name, source-target distance in meters).
\end{document}