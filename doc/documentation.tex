\documentclass[a4paper, spanish]{article}
\usepackage[utf8x]{inputenc}
\usepackage{ucs}
\usepackage{amsmath}
\usepackage{amsfonts}
\usepackage{amssymb}
\usepackage{graphicx}
\usepackage{listings}
\usepackage{color, colortbl}
\usepackage{geometry}
\usepackage{indentfirst}
\usepackage[colorlinks=true,linkcolor=blue,urlcolor=blue]{hyperref}
\usepackage{enumitem}
\usepackage{amsmath}

\definecolor{codegreen}{rgb}{0,0.6,0}
\definecolor{codegray}{rgb}{0.5,0.5,0.5}
\definecolor{codepurple}{rgb}{0.58,0,0.82}
\definecolor{backcolour}{rgb}{0.95,0.95,0.92}
\definecolor{BLUE}{rgb}{0.0, 0.75, 1.0}

\lstdefinestyle{mystyle}{
	backgroundcolor=\color{backcolour},   
	commentstyle=\color{codegreen},
	keywordstyle=\color{magenta},
	numberstyle=\tiny\color{codegray},
	stringstyle=\color{codepurple},
	basicstyle=\footnotesize,
	breakatwhitespace=false,         
	breaklines=true,                 
	captionpos=b,                    
	keepspaces=true,                 
	numbers=left,                    
	numbersep=5pt,                  
	showspaces=false,                
	showstringspaces=false,
	showtabs=false,                  
	tabsize=2
}

\date{\textbf{\url{https://github.com/thmasker/A1-06}}}
\author{Pablo Alcázar Morales \and Güray Meriç \and Diego Pedregal Hidalgo}
\title{\textbf{IS Lab Project \textbar A1-06}}
\begin{document}
	\maketitle
	\section{\textit{Task1}}
	For this task, we found a simple, effective and easy to use library on Python, called  \href{https://docs.python.org/3/library/xml.etree.elementtree.html#module-xml.etree.ElementTree}{\textit{\underline{ElementTree}}}, so after seeing other options like \textit{minidom}, we decided to use it.
	After that, we created the required \textit{Graph} class, which constructor parses the chosen \textit{.graphml} file, and stores both nodes and edges in two independent variables. Both \textit{belongNode} and \textit{positionNode} methods are really easy to understand, so we won't go on further explanations.
	
	
	The most complex of the three methods created is \textit{adjacentNode} (see \textit{Figure 1.1}). This method goes through all the node's edges, and appends to the adjacency list those which source node is the \textit{nodeid} we are looking for. Firstly, we append each edge to the adjacency list with their \textit{name} and \textit{length} fields as \textit{SinNombre} and \textit{NULL} respectively. After that we search both fields on the inner data of each edge using the keys dictionary created on the constructor, and we refill both attributes with the real ones. If anyone of them is missing, we'll know that, as it will appear as \textit{SinNombre} or \textit{NULL} in the adjacency list returned.
	
	\lstset{style=mystyle}
	\lstinputlisting[title={Figure 1.1 Method \textit{adjacentNode}},language=python]{"code fragments/adjacentNode.txt"}
	\lstset{style=mystyle}
	\lstinputlisting[title={Figure 1.2 Keys dictionary}, firstline=25, lastline=27]{"code fragments/adjacentNode.txt"}
	
	
		\subsection{Some results}
		We coded a script (\textit{tests.py}) to be able to test the results given by the program. We show some of them next:
		\lstset{style=mystyle}
		\lstinputlisting[title={Figure 1.1.1 Result for \textit{nodeid} 1851273807 from \textit{Abenójar.graphml}}, firstline=1, lastline=3]{"code fragments/results_adjacentNode.txt"}
		\lstset{style=mystyle}
		\lstinputlisting[title={Figure 1.1.2 Result for \textit{nodeid} 764039207 from \textit{Ciudad Real.graphml}}, firstline=5, lastline=7]{"code fragments/results_adjacentNode.txt"}
		\lstset{style=mystyle}
		\lstinputlisting[title={Figure 1.1.3 Result for \textit{nodeid} 2012263661 from \textit{Anchuras.graphml}}, firstline=9, lastline=11]{"code fragments/results_adjacentNode.txt"}
		
		\subsection{Timing}
		\lstset{style=mystyle}
		\lstinputlisting[title={Figure 1.2.1 Result for \textit{nodeid} 4753226234 from \textit{Abenójar.graphml}}, firstline=13, lastline=14]{"code fragments/results_adjacentNode.txt"}
		\lstset{style=mystyle}
		\lstinputlisting[title={Figure 1.2.2 Result for \textit{nodeid} 764039166 from \textit{Ciudad Real.graphml}}, firstline=16, lastline=17]{"code fragments/results_adjacentNode.txt"}
		\lstset{style=mystyle}
		\lstinputlisting[title={Figure 1.2.3 Result for \textit{nodeid} 4928063639 from \textit{Anchuras.graphml}}, firstline=19, lastline=20]{"code fragments/results_adjacentNode.txt"}
	
	\section{\textit{Task2}}
	On this task, we implemented five different classes:
		\begin{enumerate}[leftmargin=2cm]
			\item \textit{TreeNode}. This class has not anything special, but the only required attributes: \textit{parent, state, cost of the path, action, current depth} and \textit{f}, which is a random number between 1 and 10 000.
			\item \textit{State}. As we should not consider that the list of nodes given will be already ordered, we implemented the typical \textit{mergeSort} algorithm to order the elements in an optimized way, which running time in asymptotic notation is:
				\begin{equation}
					O(n\log n)
				\end{equation}
			\lstset{style=mystyle}
			\lstinputlisting[title={Figure 2.1 \textit{mergeSort} implementations}, language=python]{"code fragments/mergesort_State.txt"}
			\item \textit{StateSpace}. This class basically generates the successors for every specific \textit{State} on the problem. Obviously, the main functionality is the function \textit{successors}, which inserts in a list the action to do, its cost, and the new \textit{State} caused for each adjacent node of the current state.
			\lstset{style=mystyle}
			\lstinputlisting[title={Figure 2.2 \textit{successors} implementation}, language=python]{"code fragments/successors_StateSpace.txt"}
			\item \textit{Problem}. In this case, there is not much difficulty, as we only read the initial state of the problem from a \textit{.json} file (thanks to \textit{json} library). We also created the method \textit{isGoal}, which returns \textit{True} if the goal has been reached and \textit{False} otherwise.
			\item \textit{Frontier}.
		\end{enumerate}
	

\end{document}